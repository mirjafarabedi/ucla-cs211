\begin{abstract}
OpenWRT is a publicly available Linux distribution for various routers and embedded devices, i.e., a router operating system. Current OpenWRT remote access is handled through ssh command lines or pre-packaged LuCI web interface. However, both of them have limitations. SSH is designed for experts but not normal users, and it does not provide graphical UI. LucI Web interface is not touch-friendly and its functionalities are limited to what is provided by OpenWRT. To overcome the limitations, in this project, we designed and implemented an Android application for configuring OpenWRT APs, reused the LuCI web interface to use its exising functionalities, and designed and implemente a Python backend on the AP to extend the functionalities provided by LuCI web interface. We provided demonstrations for the initial frontend application, the UI designs for future versions of the application, and basic profiling results of the per-device per-application statistics backend. Experiments on both virtual machines and phisical machines show that our design and implementation overcomes the limitations of SSH and LuCI web interface, withouting consuming too much resources of the machines.
\end{abstract}