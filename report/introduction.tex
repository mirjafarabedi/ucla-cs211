\section{Introduction}
	
OpenWRT is a free, open-source, Linux-kernel based operating system (OS) for network-routing embedded systems. This OS is notable for being able to run on various types of devices and for simplifying cross-platform building of OpenWRT software packages, including fixes for devices no longer supported by the devices' manufacturers. OpenWRT receives regular updates, and allows basic router configuration, and installation of features through a package repository.

OpenWRT can be configured using a SSH command-line interface or a pre-packaged LuCI web interface. The SSH command-line interface is more suitable to developers, while a web interface is more friendly to common users, providing access to basic OpenWRT functions. However, SSH is tedious on a mobile device, because the user must perform all configurations via text modifications, and data and results cannot be interpreted in visual graphics. On the other hand, OpenWRT's existing remote access web interface is made for desktop web browsers, and not smart devices. On mobile smart devices, the in-browser interface is not scaled to the dimensions of the device's screen nor is it touch-friendly, hampering user comprehension and control of the interface. Additionally, depending on the device, loss of connection or suspension of the browser to another application can force renewal of the session or prevent retrieval of network information, causing issues for presenting real-time data and visualizations.
	
A native smart device application is more distribution-friendly and user-friendly. Therefore, this project seeks to create a lightweight, easy-to-use generic Android application, which allows easy monitoring and configuration for products running OpenWRT. In addition to the basic configuration functions that are expected from a typical AP, this project also includes an application-specific traffic statistics tool running on OpenWRT, which analyzes OpenWRT's incoming and outgoing traffic, and produces statistics on how much traffic's generated per-IP address per-application. Such statistics are visualized on the Android application as well.