\subsection{Backend implementation}

\subsubsection{LuCI Web Interface setup}
We did not modify the LuCI Web Interface. We used ``opkg'' package managing tools to install LuCI on OpenWRT machine. To use LuCI, a very important thing is that the firewall should be set correctly, so that LuCI can provide service on port 80. It is possible and easy to add new function to LuCI web interface, however, we do not find useful functions to add, so we just keep it as it is.

\subsubsection{Per-application statistics analysis backend implementation}

Based on the design described in section \ref{sec:app-specific-design}, the proof of concept implementation pipelines the output of \textit{tcpdump} to a custom Python backend. The backend then reads the source and destination IP addresses, tries to use \textit{nslookup} to find domain names by DNS reverse lookup, and matches the domain names with statically configured characteristic strings. If a match is found, the number of bytes is added for corresponding service provider name and corresponding host; otherwise, the traffic is considered identified. The backend writes statistics in JSON format to a file, and the file is served by a simple web server. In the statistics, incoming and outgoing traffic, as well as different IP addresses within the local subnet, is differentiated. An example of the JSON object is given below, in which the numbers are number of bytes per IP address and application. We use Python for the simple backend and web server, for easy prototyping purposes.

\begin{listing}
	\begin{minted}[frame=none,
	framesep=3mm,
	linenos=true,
	xleftmargin=21pt,
	tabsize=4]{js}
	{"outgoing": 
	{"google": {"10.0.5.15": 812}, 
	"facebook": {"10.0.5.15": 622}}, 
	"incoming": 
	{"google": {"10.0.5.15": 148865}, 
	"facebook": {"10.0.5.15": 76947}}
	}
	\end{minted}
\end{listing}

Basic optimizations for improving the performance of the backend include introducing the following:
\begin{itemize}
	
	\item A cache for IP and domain mapping, so that instead of calling \textit{nslookup} each time a packet arrives, \textit{nslookup} is tried only once for each IP address within a given time.
	
	\item A minimum waiting interval of one minute between writing statistics back, so that the number of file writes is limited.
	
\end{itemize}