\section{Design and implementation \\ roadmap}
	
Based on the functional modules of LuCI, we will design the functional components of the Android application, which are organized into three major categories:
	
\begin{itemize}
		
		\item
		\textbf{Network configuration} covers common functionalities in configuration tools that often come with commercial APs. These configurations include, but may not be limited to, managing network interfaces, DHCP and DNS settings, static routes, and firewall.
		
		\item
		\textbf{System configuration} provides an interface to customize the OpenWRT box. Common adminstration functions include: system and user configuration (setting device adminstrator password, creating system backup image and restoring system from backup image, generating user SSH keys, etc), software management (installing and configuring software packages) and task management (managing scheduled task and startup task). If the time allows, an in-application command line tool can be implemented for advanced users to execute console commands from the application to further customize the OpenWRT box.
		
		\item
		\textbf{Status/Statistics visualization} offers a mobile-phone friendly view of the system status (firmware and kernel version, uptime, current time; CPU and memory usage, currently running processes, system and kernel log) and network-related status (interface, route, firewall status, etc). The visualization component can provide real-time graphs of system load and traffic statistics, such as historical system memory usage, network traffic per interface and traffic per transport layer connection.
		
\end{itemize}
	
	The design and implementation effort will be organized by the three function categories, with approximately one and a half weeks dedicated to each.
	
	\subsection{Graphical User Interface}
	The goal of the graphical user interface design was to simplify the user interface on a smartphone. To that end, the limitations of the LuCI web interface were studied to provide design guidelines. The first issue analyzed was that LuCI had an issue in its navigation on smartphones. To navigate through the application categories, the user needed to select a category in the navigation menu, then select a subcategory from the dropdown menu. Additionally, changing subcategories within the same category still required selecting the overarching category again. Therefore, a design goal of the application would be to maintain the current category and simply swap subcategories.
	
	Another LuCI WebView issue was that unsaved changes would be tracked in a session until committed. Tracking unsaved changes on a web browser can result in session complications, depending on browser settings for caching. Furthermore, the WebView relied on in-browser scripting to provide functional elements, which is a dependency that can be optimized. Therefore another aspect of our design was to make all actions atomic and contained to the screen they are accessed on, to avoid carrying changes. To make all actions atomic, all functional elements in the original WebView would be rebuilt natively in Android.
	
	Based on the LuCI framework, the designed Android application's user interface screens consisted of a login screen, then three major categories: status, network, and system. Each category then presented a subnavigation menu that persisted until another major category was selected, allowing users to move more freely within same category.
	
	Each separate screen in the subcategories of the major categories was designed to maintain discrete actions and information. Rather than having multiple configuration forms and submission buttons in the same screen, the screen would be limited to at most one form each, with other forms being accessible on a new screen that is linked to by a list on the current screen.


% \begin{figure}
% 	\centering
% 	\includegraphics[width=0.4\textwidth]{}
% 	\caption{Workflow of the app specific traffic statistics backend}
% 	\label{fig:python-backend}
% \end{figure}