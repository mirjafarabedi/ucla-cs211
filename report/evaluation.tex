\section{Evaluation}
To evaluate our implementation, we test the backend on both a virtualBox machine and a phsical OpenWRT router; we also test the Android applcation on both Android Studio built-in smartphone emulation and physical Android devices. Here are some details of the evaluation.

\subsection{Application-specific traffic statistics backend performance evaluation}

This section describes the disk, CPU and memory usage of the Python backend for application specific traffic analysis. 

The hard drive volume is usually quite limited for an OpenWRT box. The backend analyzer had dependencies on opkg packages \textit{tcpdump}, \textit{python-light} and \textit{python-codecs}; and the associated HTTP server had dependencies on \textit{python-logging} and \textit{python-ssl} (optional, the module would not give out warnings if installed). On our test TP-Link WDR4300 router flashed with OpenWRT 15.05, 500KB disk space still remained for the mount point ``/'', after all the necessary dependencies were installed. The mount point ``/tmp'' had over 60MB temporary space, and we stored our tcpdump and statistics output files under mount point ``/tmp''.

CPU and memory usage on the OpenWRT boxes were measured on the activity of users using the AP to watch YouTube videos. When only one user was using the AP to watch a Youtube video at 360p, our Python backend used 8\%~22\% CPU, and tcpdump used 1\%~10\% depending on whether the video was pre-buffered long enough. Both tcpdump and Python backend used ~5\% of the virtual memory. With two users using the AP, the Python backend CPU usage grew by 7\% on average, and we did not see tcpdump CPU usage ramp up. Since the communication between tcpdump and the Python backend is through file system interface, I/O was the more likely than CPU to be the culprit behind the bottleneck.

\subsection{Demo description}

A screen recording of the current Android application implementation is available at \url{http://memoria.ndn.ucla.edu/openwrt2.mp4}. This demo used a TP-Link WDR4300 router with LuCI and the application specific analysis backend installed. In the demo, the phone application first configures the SSID of the router from ``OpenWRT'' to ``OpenWRT123'', and reconnects to WiFi using the changed SSID. Then the phone generates Youtube traffic by visiting its webpage, and demonstrates that the incoming traffic labeled ``Google'' grows by 4MB correspondingly. While the previous demo used WebView to browse and configure the wireless AP, another screen recording available at \url{http://memoria.ndn.ucla.edu/openwrt1.mp4} demonstrates our initial attempts of moving towards JSON and html parsing, using native Java code instead of executing JS in the WebView, and using Android UI widgets instead of relying on the WebView CSS for styling.