\section{Evaluation}


\subsection{Application specific traffic statistics backend evaluation}

This section describes the disk, CPU and memory usage of the Python backend for application specific traffic analysis. 

Hard drive volume's usually quite limited for an OpenWRT box. The backend analyzer has dependency on opkg packages \textit{tcpdump}, \textit{python-light} and \textit{python-codecs}; and the associated http server has dependency on \textit{python-logging} and \textit{python-ssl} (optional, the module would not give out warnings if installed). On our test TP-link WDR4300 router flashed with OpenWRT 15.05, 500KB disk space still remains for mount point ``/'', after all the necessary dependencies are installed. The mount point ``/tmp'' has over 60MB temporary space, and we store our tcpdump file, and statistics output file under mount point ``/tmp''.

CPU and memory usage on the OpenWRT boxes's measured with users behind the AP watching a Youtube video. When only one user's behind the AP watching a Youtube video at 360p, our Python backend uses 8\%~22\% CPU, tcpdump uses 1\%~10\% depending on whether the video is pre-buffered long enough. Both tcpdump and Python backend use ~5\% of the virtual memory. With two users behind the AP, the Python backend CPU usage grows by 7\% on average, and we did not see tcpdump CPU usage ramp up. Since the communication between tcpdump and the Python backend is through file system interface, IO is more likely the bottleneck here than CPU.
