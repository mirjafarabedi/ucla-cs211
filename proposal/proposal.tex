\documentclass{sig-alternate-05-2015}

% Do not include ISBN and DOI info
\makeatletter
\def\@copyrightspace{\relax}
\makeatother

\begin{document}

\title{Course project proposal: Remote access to OpenWRT}

\numberofauthors{3}

\author{
\alignauthor
Zhehao Wang \\
  \email{zhehao@cs.ucla.edu}
% 2nd. author
\alignauthor
Haitao Zhang
% 3rd. author
\alignauthor 
Jeffrey Chen
}

\maketitle

\section{Design and implementation roadmap}

This section covers the functional components of the application, which are organized into three major categories.

\begin{itemize}

\item
\textbf{Network configuration} covers common functionalities in configuration tools that often come with commercial APs. These configurations include, but may not be limited to, managing network interfaces, DHCP and DNS settings, static routes, and firewall.

\item
\textbf{System configuration} provides interface to customize the OpenWRT box. Common adminstration functions include: system and user configuration (setting device adminstrator password, creating system backup image and restoring system from backup image, generating user SSH keys, etc), software management (installing and configuring software packages), task management (managing scheduled task and startup task)

\item
\textbf{Status/Statistics visualization} offers a mobile-phone friendly view of the system status (Firmware and kernel version, uptime, current time; CPU and memory usage, currently running processes, and system and kernel log) and network-related status (Interface, route, and firewall status, etc). The visualization component could provide realtime graphs of system load and traffic statistics, for example, traffic per interface and traffic per transport layer connection.

\end{itemize}

The design and implementation effort will be organized by the three function categories, with approximately two weeks dedicated to each.

\section{Timeline}

A rough timeline for the project is given in table \ref{table:timeline}

\begin{table}[h]
\centering
\caption{Project timeline}
\label{table:timeline}
\begin{tabular}{c|c} \hline
Week No. & Task \\ \hline
5, 6 & Status/Statistics visualization \\ \hline
7, 8 & Network configuration \\ \hline
9, 10 & System configuration \\
\hline\end{tabular}
\end{table}

\end{document}